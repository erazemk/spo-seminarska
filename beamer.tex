% \documentclass{beamer}
\documentclass[aspectratio=169]{beamer}
\usepackage[utf8]{inputenc}

\title{Implementacije posodobitev operacijskih sistemov}
\author{Erazem Kokot}
\institute[FRI]{Sistemska programska oprema,\\Fakulteta za računalništvo in informatiko}
\date{2021/22}

\usetheme{Copenhagen}
\usecolortheme{beaver}
\setbeamertemplate{navigation symbols}{}

\begin{document}

    \begin{frame}
    \titlepage{}
    \end{frame}

    \begin{frame}
        \frametitle{Posodobitve operacijskih sistemov}
        \begin{itemize}
            \item obvezno za vse spletno povezane naprave,
            \item varnostne posodobitve, nova funkcionalnost,
            \item za sistemske komponente skrbi OS,
            \item za aplikacije skrbi OS ali pa same implementirajo posodobitve.
        \end{itemize}
    \end{frame}

    \begin{frame}
        \frametitle{Podobnosti med implementacijami}
        \begin{itemize}
            \item namestitev posodobitev preko spleta:
            \begin{itemize}
                \item hitrejše,
                \item bolj skalabilno (avtomatizacija).
            \end{itemize}
            \item brez povezave preko zunanjega media (CD/USB):
            \begin{itemize}
                \item varnejše (banke, borze, \ldots),
                \item uporaba kompresijskih algoritmov.
            \end{itemize}
        \end{itemize}
    \end{frame}

    \begin{frame}
        \frametitle{Visokonivojski postopek nameščanja}
        \begin{enumerate}
            \item kontaktiranje strežnika s posodobitvami,
            \item iskanje posodobitev,
            \item prenos najdenih posodobitev,
            \item dekompresiranje in obdelava prenosa,
            \item namestitev v sistem.
        \end{enumerate}
    \end{frame}

    \begin{frame}
        \frametitle{Razlike med implementacijami}
        \begin{itemize}
            \item kdo gostuje strežnike (centralizirani ali decentralizirani),
            \item vrsta in oblika posodobitev,
            \item razlikovanje med sistemsko in aplikacijsko programsko opremo,
            \item kompresijski algoritmi.
        \end{itemize}
    \end{frame}

    \begin{frame}
        \frametitle{Windows}
        \begin{itemize}
            \item sistemske programe posodablja OS, aplikacije se posodabljajo posebej,
            \item strežniki so v Microsoftovi lasti,
            \item za posodobitve skrbi \emph{Orkestrator},
            \item naključen čas iskanja posodobitev (obremenjenost strežnikov),
            \item samo kumulativne posodobitve (prenos celotne vsebine),
            \item \emph{forward and reverse} diferencialna kompresija (do 40 \% prihranka).
        \end{itemize}
    \end{frame}

    \begin{frame}
        \frametitle{Windows (podroben pogled)}
        \begin{itemize}
            \item delitev posodobitev na storitve (Windows Update, Microsoft Update, Store, \ldots),
            \item uporaba \emph{Delivery Optimization} za preprečevanje zasičenosti omrežja,
            \item \emph{Arbiter} uporabi metapodatke za generiranje \emph{seznama dejanj},
            \item namestitveni program uporabi seznam dejanj za namestitev,
            \item \emph{Orkestrator} ponovno zažene posodobljeno napravo.
        \end{itemize}
    \end{frame}

    \begin{frame}
        \frametitle{Linux}
        \begin{itemize}
            \item veliko distribucij \( \Rightarrow \) veliko implementacij,
            \item programi distributirani v obliki programskih paketov,
            \item jedro distributirano kot programski paket,
            \item za posodobitve skrbi \emph{package manager} (upravljalec paketov):
            \begin{itemize}
                \item primeri: DNF, APT, pacman, Zypper, \ldots
                \item naloge: iskanje, prenos in namestitev paketov, razreševanje odvisnosti, \ldots
            \end{itemize}
            \item ni razlike med sistemskimi in aplikacijskimi posodobitvami,
            \item prosto dodajanje repozitorijev,
            \item najpogostejši kompresijski algoritem je \emph{zstd (zstandard)}.
        \end{itemize}
    \end{frame}

    \begin{frame}
        \frametitle{Linux (podroben pogled): Fedora}
        \begin{itemize}
            \item upravljalec paketov DNF (wrapper za RPM),
            \item format RPM,
            \item binarna in izvornokodna oblika,
            \item binarna vsebuje zgrajeno aplikacijo ali knjižnico,
            \item izvornokodna vsebuje skripte in datoteke za grajenje binarne,
            \item format:
            \begin{itemize}
                \item podpis (kriptografski ali zgoščevalni),
                \item glava (metapodatki: licenca, verzija, opis),
                \item vsebina (\emph{payload}: kompresiran binarni program ali knjižnica),
            \end{itemize}
            \item prednost da lahko prebere metapodatke brez dekompresiranja vsebine.
        \end{itemize}
    \end{frame}

    \begin{frame}[plain,c]
       \begin{center}
            Vprašanja?
       \end{center}
    \end{frame}

\end{document}