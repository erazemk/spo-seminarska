\section{Zaključek}

V svoji seminarski nalogi sem želel podrobneje predstaviti različne postopke posodabljanja modernih operacijskih sistemov
in prikazati, v čem so si implementacije podobne ter v čem se razlikujejo.
Razvidno je, da si kljub množici implementacij le-te vseeno delijo veliko lastnosti in da si pogosto izposojajo
ideje ena od druge.
Menim, da je koncept uporabe programskih paketov za vse komponente operacijskega sistema in uporabniške programe,
ki ga uporabljajo distribucije Linux, boljši od implementacije, ki jo uporablja Windows, saj omogoča, da so vsi
programi, sistemski ali uporabniški, dostopni v isti obliki in iz istih virov, kar je boljše za varnost uporabnikov in
stabilnost sistema.
Incidentno je kvaliteta upravljalcev paketov za operacijski sistem Linux zaradi večih neodvisnih implementacij,
ki si med sabo delijo znanje in izkušnje, veliko višja.
Prednosti se pokažejo tudi pri hitrosti razvoja, saj si v nasprotju z operacijskim sistemom Windows, ki mora dosledno
skrbeti za vzvratno kompatibilnost, distribucije Linux lahko privoščijo večje spremembe in s tem tudi večjo
učinkovitost ter hitrejše delovanje upravljalcev programskih paketov.