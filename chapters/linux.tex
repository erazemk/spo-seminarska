\subsection{Linux}

Ker Linux sam po sebi ni operacijski sistem temveč le jedro sistema, obstaja veliko implementacij operacijskih sistemov,
ki temeljijo na Linux jedru oz. tako imenovanih distribucij Linux.
Kljub pomembnosti jedra Linux je le-to v večini distribucij v obliki preprostega programskega paketa,
tako da se lahko enostavno posodablja skupaj z drugimi programskimi paketi v sistemu.

Distribucije Linux za posodabljanje sistema in aplikacij uporabljajo upravljalce\linebreak programskih paketov
(ang. \emph{package manager}), katerih naloga je, da programske pakete prenesejo,
dekompresirajo in namestijo na določeno mesto v sistemu.
Implementacije upravljalcev programskih paketov se razlikujejo med distribucijami,
a si jih ponavadi distribucije, ki temeljijo na drugih, z njimi delijo.

V nasprotju z operacijskim sistemom Windows, kjer so vsi programski repozitoriji v lasti podjetja Microsoft,
distribucije Linux podpirajo prosto dodajanje uporabniških repozitorijev, nekatere pa le-te uporabljajo tudi
kot privzete vire posodobitev \cite{arch-mirrors}.\\

Ker so si distribucije med sabo tako podobne, je dovolj, da podrobno prikažemo delovanje le pri eni izmed njih, za
namene seminarske naloge sem izbral distribucijo Fedora, saj sem z njo tudi najbolj seznanjen.

\subsubsection{Visokonivojski pogled}

\href{https://getfedora.org/}{Fedora} (tudi \emph{Fedora Linux}) za upravljanje programskih paketov uporablja
orodje DNF, programski paketi pa so v formatu RPM \cite{rpm}.
Le-ti so na voljo v dveh oblikah, binarni (ang. \emph{Binary RPM}) in izvornokodni (ang. \emph{Source RPM}).
Programski paketi prve oblike vsebujejo celotno zgrajeno aplikacijo ali sistemsko knjižnico za posamezno
računalniško arhitekturo, paketi druge oblike pa vsebujejo skripte in druge datoteke, ki so potrebne za grajenje
binarne oblike programskega paketa.
Ime \emph{izvornokodni format} se ne navezuje na licenco programa, ki ga distribuiramo, saj vsebuje le navodila
za namestitev programa, kar lahko vključuje tudi le prenos že zgrajenega zaprtokodnega programa
iz določenega spletnega naslova.

\subsubsection{Format RPM}

Format RPM je sestavljen iz podpisa (ang. \emph{signature}), glave (ang. \emph{header}) in vsebine
(ang. \emph{payload}).
Podpis je lahko kriptografski (npr. PGP) ali pa zgoščevalni (npr. SHA256) in je namenjen preverjanju integritete
programskega paketa, glava pa vsebuje metapodatke o programskem paketu kot so licenca, verzija in opis.
Vsebina vključuje binarno (prevedeno) obliko programa, zaradi velikosti pa je kompresirana
s kompresijskim algoritmom zstd \cite{fedora-zstd}.

Prednost formata RPM pred drugimi formati je ta, da lahko upravljalec paketov prebere vse metapodatke paketa in preveri
njegovo integriteto, ne da bi potreboval dekompresirati vsebino paketa, kar zelo pospeši iskanje paketov in omogoča
kompleksne operacije, kot je iskanje po opisih ali pa odvisnostih.

\newpage

\subsubsection{DNF upravljalec programskih paketov}

DNF, kot tudi upravljalci programskih paketov v drugih distribucijah, omogoča veliko funkcionalnosti,
med drugim iskanje informacij o programskih paketih, prenašanje programskih paketov iz repozitorijev,
namestitev in odmestitev programskih paketov, avtomatsko razreševanje odvisnosti in celo popolno nadgradnjo sistema
iz ene verzije v drugo \cite{fedora-dnf}.\\

Če podrobno pogledamo delovanje DNF, lahko vidimo, da za upravljanje programskih paketov uporablja
\href{http://rpm.org/}{RPM} - nizkonivojski upravljalec programskih paketov, ki zna delati s formatom RPM,
pa tudi knjižnici \href{https://github.com/openSUSE/libsolv}{libsolv} in
\href{https://github.com/rpm-software-management/libdnf}{libdnf}, za delo z metapodatki in prenose programskih paketov
uporablja knjižnico \href{https://github.com/rpm-software-management/librepo}{librepo}, za procesiranje podatkov o
združevanju programskih paketov pa knjižnico \href{https://github.com/rpm-software-management/libcomps}{libcomps}
\cite{fedora-dnf-git-repo}.

Knjižnica \emph{librepo} ponuja API, ki omogoča prenos metapodatkov in programskih paketov iz Linux repozitorijev,
knjižnica \emph{libsolv} skrbi za razreševanje odvisnosti med programskimi paketi,
knjižnica \emph{libdnf} združuje funkcionalnosti librepo in libsolv in jih na visokonivojski način ponuja DNF,
knjižnica \emph{libcomps} pa je namenjena upravljanju metapodatkov o skupinah programskih paketov, ki jih DNF uporabi
za množično nameščanje podobnih programskih paketov (npr. programi, ki so potrebni za grafično okolje GNOME).

\subsubsection{Upravljalci programskih paketov v drugih distribucijah Linux}

Ostali znani upravljalci programskih paketov drugih distribucij so
\href{https://ubuntu.com/server/docs/package-management}{apt}, uporabljen v distribucijah
\href{https://www.debian.org/}{Debian}, \href{https://ubuntu.com/}{Ubuntu} in njunih izpeljavah (Ubuntu je izpeljan iz
Debian, a je zaradi pogostosti uporabe vreden omembe), \href{https://wiki.archlinux.org/title/pacman}{\emph{pacman}},
uporabljen v distribuciji \href{https://archlinux.org/}{Arch Linux} in njegovih izpeljavah in
\href{https://en.opensuse.org/Portal:Zypper}{\emph{zypper}}, uporabljen v distribuciji
\href{https://www.opensuse.org/}{OpenSUSE}.