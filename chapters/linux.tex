\subsection{Linux}

Ker Linux sam po sebi ni operacijski sistem temveč le jedro sistema, obstaja veliko implementacij operacijskih sistemov,
ki temeljijo na Linux jedru oz. tako imenovanih distribucij Linux.
Kljub pomembnosti jedra Linux je le-to v večini distribucij v obliki preprostega paketa, tako da se lahko enostavno
posodablja skupaj s drugimi paketi v sistemu.

Distribucije Linux za posodabljanje sistema in aplikacij uporabljajo upravljalce paketov (ang. \emph{package manager}),
katerih naloga je, da pakete prenesejo, dekompresirajo in namestijo na določeno mesto v sistemu.
Implementacije upravljalcev paketov se razlikujejo med distribucijami, a si jih ponavadi distribucije, ki temeljijo na
drugih z njimi delijo.

Ker so si distribucije med sabo tako podobne, je dovolj, da podrobno prikažemo delovanje le pri eni izmed njih, nato
pa pri ostalih omenimo le razlike - za namene seminarske naloge sem izbral distribucijo Fedora, saj sem z njo tudi
najbolj seznanjen.

% Notes:
%   - sistem se posodablja s package manager-om
%   - "sistem" pomeni kernel + paketi
%   - packagei so v repozitoriju
%   - opiši Ubuntu (apt), Fedoro (dnf) in Arch (pacman)
%   - opiši compression algoritem
%   - opiši postopek inštalacije paketa


\subsubsection{Upravljalec paketov}

Fedora (tudi \emph{Fedora Linux}) za upravljanje paketov uporablja orodje DNF, paketi pa so v formatu RPM.

Le-ti so na voljo v dveh formatih, binarnem (ang. \emph{Binary RPM}) in izvornokodnem (ang. \emph{Source RPM}).
Paketi prvega formata vsebujejo celotno zgrajeno aplikacijo ali sistemsko knjižnico za posamezno
računalniško arhitekturo, paketi drugega formata pa vsebujejo skripte in druge datoteke, ki so potrebne za grajenje
binarnega formata paketa.
Ime \emph{izvornikodni format} se ne navezuje na licenco programa, ki ga distributiramo, saj vsebuje le navodila
za namestitev programa, npr. prenos že zgrajenega programa iz določenega spletnega naslova.

Format RPM je sestavljen iz podpisa, glave (ang. \emph{header}) in vsebine (ang. \emph{payload}).
Podpis je lahko kriptografski (npr. PGP) ali pa zgoščevalni (npr. SHA256) in je namenjen preverjanju integritete paketa,
glava pa vsebuje metapodatke o paketu kot so npr. licenca, verzija in opis.
Vsebina vključuje celotno binarno kodo programa, zaradi velikosti pa je kompresirana s kompresijskim algoritmom zstd.

Prednost formata RPM pred drugimi formati je ta, da lahko upravljalec paketov prebere vse metapodatke paketa in preveri
njegovo integriteto, ne da bi potreboval dekompresirati vsebino paketa, kar zelo pospeši iskanje paketov in omogoča
kompleksne operacije, kot je iskanje po opisih ali pa odvisnostih.