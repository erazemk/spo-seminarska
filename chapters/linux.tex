\subsection{Linux}

Ker Linux sam po sebi ni operacijski sistem temveč le jedro sistema, obstaja veliko implementacij operacijskih sistemov,
ki temeljijo na Linux jedru oz. tako imenovanih distribucij Linux.
Kljub pomembnosti jedra Linux je le-to v večini distribucij v obliki preprostega programskega paketa,
tako da se lahko enostavno posodablja skupaj s drugimi programskimi paketi v sistemu.

Distribucije Linux za posodabljanje sistema in aplikacij uporabljajo upravljalce\linebreak programskih paketov
(ang. \emph{package manager}), katerih naloga je, da programske pakete prenesejo,
dekompresirajo in namestijo na določeno mesto v sistemu.
Implementacije upravljalcev programskih paketov se razlikujejo med distribucijami,
a si jih ponavadi distribucije, ki temeljijo na drugih, z njimi delijo.

V nasprotju z operacijsim sistemov Windows, kjer so vsi programski repozitoriji v lasti Microsoft,
distribucije Linux podpirajo prosto dodajanje uporabniških repozitorijev, nekatere pa le-te uporabljajo tudi
kot privzete vire posodobitev \cite{arch-mirrors}.

Ker so si distribucije med sabo tako podobne je dovolj, da podrobno prikažemo delovanje le pri eni izmed njih, nato
pa pri ostalih omenimo le razlike - za namene seminarske naloge sem izbral distribucijo Fedora, saj sem z njo tudi
najbolj seznanjen.\\

Fedora (tudi \emph{Fedora Linux}) za upravljanje programskih paketov uporablja orodje DNF,
programski paketi pa so v formatu RPM \cite{rpm}.

Le-ti so na voljo v dveh oblikah, binarni (ang. \emph{Binary RPM}) in izvornokodni (ang. \emph{Source RPM}).
Programski paketi prve oblike vsebujejo celotno zgrajeno aplikacijo ali sistemsko knjižnico za posamezno
računalniško arhitekturo, paketi druge oblike pa vsebujejo skripte in druge datoteke, ki so potrebne za grajenje
binarne oblike programskega paketa.
Ime \emph{izvornikodni format} se ne navezuje na licenco programa, ki ga distributiramo, saj vsebuje le navodila
za namestitev programa, kar lahko vključuje tudi le prenos že zgrajenega zaprtokodnega programa
iz določenega spletnega naslova.

Format RPM je sestavljen iz podpisa (ang. \emph{signature}), glave (ang. \emph{header}) in vsebine
(ang. \emph{payload}).
Podpis je lahko kriptografski (npr. PGP) ali pa zgoščevalni (npr. SHA256) in je namenjen preverjanju integritete
programskega paketa, glava pa vsebuje metapodatke o programskem paketu kot so licenca, verzija in opis.
Vsebina vključuje binarno (prevedeno) obliko programa, zaradi velikosti pa je kompresirana
s kompresijskim algoritmom zstd \cite{fedora-zstd}.

Prednost formata RPM pred drugimi formati je ta, da lahko upravljalec paketov prebere vse metapodatke paketa in preveri
njegovo integriteto, ne da bi potreboval dekompresirati vsebino paketa, kar zelo pospeši iskanje paketov in omogoča
kompleksne operacije, kot je iskanje po opisih ali pa odvisnostih.

DNF, kot tudi upravljalci programskih paketov v drugih distribucijah,
omogoča veliko funkcionalnosti, med drugim iskanje informacij o programskih paketih, prenašanje programskih
paketov iz repozitorijev, namestitev in odmestitev programskih paketov, avtomatsko razreševanje odvisnosti in celo
popolno nadgradnjo sistema iz ene verzije na drugo \cite{fedora-dnf}.\\

Ostali znani upravljalci programskih paketov drugih distribucij so \emph{APT} \cite{ubuntu-pm},
uporabljen v distribucijah Debian, Ubuntu in njunih izpeljavah (Ubuntu je izpeljan iz Debian,
a je zaradi pogostosti uporabe vreden omembe), \emph{pacman} \cite{arch-pm}, uporabljen v distribuciji Arch Linux
in njegovih izpeljavah in \emph{Zypper} \cite{opensuse-pm}, uporabljen v distribuciji OpenSUSE.