\section{Uvod}

Posodobitve operacijskih sistemov ali njihovih komponent so zaradi lahkote izvedbe napadov skoraj obvezne za vse
spletno povezane naprave.
Praktično nemogoče je zagotoviti, da bo koda, ki jo pišejo programerji, brez napak, kar pomeni, da potrebujejo
vse večje komponente, sistemske ali drugače, mehanizem posodobitev.
V primeru sistemske programske opreme, ki je tesno vezana na operacijski sistem, ta mehanizem ponuja kar operacijski
sistem, manj tesno vezane aplikacije, npr. spletni brskalniki, pa jih zagotavljajo same.

Zaradi razlik med operacijskimi sistemi, se lahko mehanizmi posodobitev zelo razlikujejo.
V naslednjih poglavjih bom predstavil nekaj skupnih lastnosti vseh mehanizmov posodobitev, nato pa podrobnosti
implementacij v dveh najpogostejših operacijskih sistemih - Windows in Linux.

\section{Skupne lastnosti implementacij}

Omenjena operacijska sistema, pa tudi množica drugih, uporabljata spletno povezavo za prenos posodobitev,
kar je dandanes precej standarden postopek, saj omogoča hitrejše in učinkovitejše nameščanje posodobitev
kot ročen prenos in namestitev preko zunanjega medija, kot je npr. USB ključek ali zgoščenka.

Prav tako oba omenjena operacijska sistema za prenos posodobitev uporabljajata kompresijske algoritme,
ki lahko bistveno zmanjšajo velikost in skrajšajo čas prenosa posodobitev.

Tudi postopek nameščanja posodobitev je podoben in sicer posodobitveni program kontaktira strežnik s posodobitvami,
pri njem preveri ali so na voljo posodobitve, le-te prenese, če so na voljo, nato pa jih dekompresira in
kakorkoli drugače obdela ter jih namesti v sistem.

\section{Razlike med implementacijami}

Kljub podobnostim med implementacijami najpogostejših operacijskih sistemov, imajo le-te tudi kar nekaj razlik,
predvsem kar se tiče distribucije in lastništva strežnikov s posodobitvami, vrst posodobitev in kompresijskih
algoritmov, ki jih uporabljajo za prenos posodobitev.